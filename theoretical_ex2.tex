\documentclass[12pt]{article}
\usepackage{color, amssymb}
\usepackage[greek,english]{babel}
\usepackage[utf8]{inputenc}
\usepackage[T1]{fontenc}  
\usepackage[dvipsnames]{xcolor}
\renewcommand{\baselinestretch}{1.5}
\usepackage{float}
\usepackage{caption}
\usepackage{graphicx}
\usepackage{epstopdf}
\usepackage{listings}
\usepackage{amsmath}

\usepackage[a4paper, total={7in, 8in}]{geometry}
\usepackage{units}
\usepackage{graphicx}

\usepackage{tikz}
\usepackage{amsmath}
\usepackage{amssymb}


\usepackage{subcaption}
\renewcommand\thesubfigure{(\roman{subfigure})}\usepackage{pgf, tikz}
\usetikzlibrary{arrows, automata}
\usepackage{pgf, tikz}
\usetikzlibrary{arrows, automata}
\usepackage{hyperref}

\newcommand{\R}{\mathbb{R}}




\begin{document}
	
		%\captionsetup[figure]{labelfont={default},labelformat={default},labelsep=period,name={Σχήμα.}}
	
	\greektext
	
	\title{ \textbf{Δεύτερη Θεωρητική Άσκηση}}
	\author{Δελατόλας Θάνος \\ AM:2016030074}
	\date{}
	
	
	\maketitle
	\newpage
	\section*{ Άσκηση 1}
	Έστω :
	\begin{itemize}
		\item \textlatin{B} : ``Ο Σταμάτης βρίσκεται στο Βερολίνο``
		\item \textlatin{M} : ``Ο Σταμάτης βρίσκεται στη Μόσχα ``
		\item \textlatin{T} : ``Ο Σταμάτης βρίσκεται στα Τρίκαλα``
		\item \textlatin{P} : ``Ο Σταμάτης φοράει 1 παλτό`` ( $\neg P$ ``Ο Σταμάτης φοράει 2 παλτά``)
	\end{itemize}
		
	\noindent
	Οι πιθανοί κόσμοι σε \textlatin{CNF} είναι:
	\begin{itemize}
		\item  $Β \wedge \neg M \wedge \neg T \wedge P$ (Ο Σταμάτης βρίσκετα στο Βερολίνο με 1 παλτό)
		\item  $Β \wedge \neg M \wedge \neg T \wedge \neg P$ (Ο Σταμάτης βρίσκετα στο Βερολίνο με 2 παλτά)
		\item  $\neg Β \wedge M \wedge \neg T \wedge P$ (Ο Σταμάτης βρίσκετα στη Μόσχα με 1 παλτό)
		\item  $\neg Β \wedge M \wedge \neg T \wedge \neg P$ (Ο Σταμάτης βρίσκετα στη Μόσχα με 2 παλτά)
		\item  $\neg Β \wedge \neg M \wedge  T \wedge P$ (Ο Σταμάτης βρίσκετα στα Τρίκαλα με 1 παλτό)
		\item  $\neg Β \wedge \neg M \wedge  T \wedge \neg P$ (Ο Σταμάτης βρίσκετα στα Τρίκαλα με 2 παλτά)
	\end{itemize}
	
	\section*{ Άσκηση 2}
	Έστω \textlatin{P} : ``Ο Γιαννάκης διψάει`` και \textlatin{Q} : ``Πίνει νερό`` τότε:
	\begin{itemize}
		\item η  πρόταση \textlatin{(i)} γίνεται: $P \rightarrow Q$
		\item η  πρόταση \textlatin{(ii)} γίνεται: $\neg Q \rightarrow \neg P$
		\item η  πρόταση \textlatin{(iii)} γίνεται: $\neg P \rightarrow \neg Q$
	\end{itemize}
	
	
	\noindent
	Που είναι λογικά ισοδύναμες (απο τη θεωρία) η καθε μια αντίστοιχα:
	\begin{itemize}
		\item η  πρόταση \textlatin{(i)} γίνεται: $\neg P \lor Q$
		\item η  πρόταση \textlatin{(ii)} γίνεται: $\neg \neg Q \lor \neg P$, δηλαδή $Q \lor \neg P$
		\item η  πρόταση \textlatin{(iii)} γίνεται: $\neg \neg P \lor \neg Q$, δηλαδή  $P \lor \neg Q$
	\end{itemize}
	
	
	\noindent
	Συμπαιραίνουμε πως οι προτάσεις \textlatin{(i),(ii)} είναι λογικά ισοδύναμες εφόσον  $\neg P \lor Q \equiv Q \lor \neg P$ .\\
	
	\noindent
	Θέλουμε τώρα να δείξουμε αν η πρόταση $(i) \rightarrow (iii)$ είναι ικανοποιήσιμη ή όχι.\\
	ή αν η πρόταση  $\neg P \lor Q \rightarrow P \lor \neg Q$ είναι ικανοποιήσιμη ή όχι.\\
	ή αν η πρόταση  $\neg (\neg P \lor Q) \lor (P \lor \neg Q)$ είναι ικανοποιήσιμη ή όχι.\\
	ή αν η πρόταση  $(\neg \neg P \wedge \neg Q) \lor (P \lor \neg Q)$ είναι ικανοποιήσιμη ή όχι.(εφαρμόστηκε νόμος του \textlatin{De Morgan})\\
	ή αν η πρόταση  $(P \wedge \neg Q) \lor (P \lor \neg Q)$ είναι ικανοποιήσιμη ή όχι.\\
	ή αν η πρόταση  $(P  \lor P \lor \neg Q) \wedge (\neg Q \lor P \lor \neg Q)$ είναι ικανοποιήσιμη ή όχι.\\
	ή αν η πρόταση  $(P  \lor \neg Q) \wedge ( P \lor \neg Q)$ είναι ικανοποιήσιμη ή όχι.\\
	ή αν η πρόταση  $ P \lor \neg Q$ είναι ικανοποιήσιμη ή όχι.\\
	
	
	\noindent
	Για να δείξουμε αν είναι ικανοποιήσιμη θα κατασκευάσουμε τον πίνακα αληθείας:	
	\begin{displaymath}
	\begin{array}{|c c|c|c|}
	% |c c|c| means that there are three columns in the table and
	% a vertical bar ’|’ will be printed on the left and right borders,
	% and between the second and the third columns.
	% The letter ’c’ means the value will be centered within the column,
	% letter ’l’, left-aligned, and ’r’, right-aligned.
	P & Q & \neg Q & P \lor \neg Q\\ % Use & to separate the columns
	\hline % Put a horizontal line between the table header and the rest.
	T & T & F & T\\
	T & F & T & T\\
	F & T & F & F\\
	F & F & T & T\\
	\end{array}
	\end{displaymath}
	
	\noindent 
	Παρατηρούμε πως υπάρχουν 3 μοντέλα που την ικανοποιούν, άρα η πρόταση $(i) \rightarrow (iii)$ είναι ικανοποιήσιμη.
	
	\section*{Άσκηση 3}
	Το πεδίο ορισμού δίνεται απο την εκφώνηση κάθε μεταβλητης και επίσης ισχυει: $\forall x,y$ $ O_{x,y} = 0 $ ή $ O_{x,y} = 1$ (ακριβώς \textlatin{k} τετράγωνα θα έχουν 1 και συνεπώς θα είναι και διαφορετικά αυτά τα τετράγωνα).\\
	Έστω \textlatin{} το σύνολο των 8 επιθετικών κινήσεων ενός ίππου,\\
	$A=\{\left[2,1\right],\left[2,-1\right],\left[-2,1\right],\left[-2,-1\right],\left[1,2\right],\left[1,-2\right],\left[-1,2\right],\left[-1,-2\right]\}$.
	
	\noindent
	Συνεπώς ο περιορισμός (\textlatin{constraint}) είναι:
	\begin{itemize}
		\item $\forall i<j$ $O_{x_i,y_i}\notin \{O_{x_j+k,y_j+l}\} | k,l \in A$ (να μην επιτίθεται ο ένας ίππος στον άλλον)
	\end{itemize}
	\noindent
	Λέμε $\forall i<j$  αντί $\forall i,j$  για να μειώσουμε τους συνολικούς περιορισμούς στη μέση. Η μείωση αυτή δεν αποτελεί πρόβλημα για την λύση του προβλήματος διότι ενας ίππος \textlatin{M} δεν μπορεί να επιτεθεί σε έναν ίππο Ν αν και μόνο αν ο Ν δεν μπορεί να του επιτεθεί.\\
	
	
	\noindent
	Τοπική αναζήτηση:
	\begin{itemize}
		\item Μια πλήρης κατάσταση είναι μια ανάθεση τιμών σε όλες τις μεταβλητές ακόμη κι αν παραβιάζονται κάποιοι περιορισμοί.
		\item Μια τοπική κίνηση είναι ανάθεση τιμής σε κάποια μεταβλητή.
		\item Συνάρτησε κόστους: +1 για κάθε μεταβλητη που δεν παραβιάζει τους περιορισμούς, -1 για κάθε μεταβλητη που τους παραβιάζει
	\end{itemize}
	\section*{Άσκηση 4}
	Ο \textlatin{Minimax} για τα παιχνίδια μηδενικού αθροίσματος (\textlatin{zero-sum games}) δουλευει όπως δουλευει για τα \textlatin{multiplayer games}. Ωστόσο, η \textlatin{evaluation function} στον Α θα συγκρίνει τη τιμή που παράγει με την τιμή που παράγει η \textlatin{evaluation function} του Β. Ο \textlatin{minimax} στον Α θα πηγαίνει σε βάθος \textlatin{d+1}. Συνεπώς για κάθε κόμβο δεν θα έχουμε μια τιμή, αλλα θα έχουμε ενα δίανυσμα δυο τιμών και θα επιστρέφεται η μεγαλύτερη τιμή.\\
	
	\noindent
	
	\section*{Άσκηση 5}
	Θα γράφουμε τα σύμβολα σχέσης στα αγγλικά. Δηλαδη αντί να γράφουμε Πατέρας(\textlatin{x,y}) θα γράφουμε \textlatin{Father(x,y)} και αντίστοιχα για τα υπόλοιπα σύμβολα σχέσης. Η δοσμένη πρόταση μπορεί να γραφτεί ως εξής: \\
	\noindent
	\[  \forall x \forall y ((\neg Father(x,y) \lor \neg Woman(x)) \wedge (\neg Mother(x,y) \lor Woman(x)) )\]
	Θεωρώντας πως υπονοούνται οι καθολικοί ποσοδείκτης ($ \forall x \forall y$) η παραπάνω πρόταση βρίσκεται σε \textlatin{CNF} μορφή και γίνεται:\\
	
	\begin{equation} \label{eq:1} \tag{5.1}
		((\neg Father(x,y) \lor \neg Woman(x)) \wedge (\neg Mother(x,y) \lor Woman(x)) )
	\end{equation}
	Θέλουμε να αποδείξουμε πως :
	\[ 
		Father(Koulhs,Alex) \rightarrow \neg Mother(Koulhs,Volfgang) 
	\]
	\noindent 
	ή 
	\[
		 \neg Father(Koulhs,Alex) \lor \neg Mother(Koulhs,Volfgang)
	\]
	\noindent
	Συμφωνα με την τεχνική της Ανάλυσης δεχόμαστη το αντίθετο απο το ζητολυμενο. Συνεπώς εισάγουμε στη βάση γνώσης τη πρόταση:
	\[
		\neg (\neg Father(Koulhs,Alex) \lor \neg Mother(Koulhs,Volfgang))
	\]
	\noindent 
	ή
	
	\begin{equation} \label{eq:2} \tag{5.2}
		 Father(Koulhs,Alex) \wedge  Mother(Koulhs,Volfgang)
	\end{equation}
	\noindent
	Με αντικατάσταση {\textlatin{ \{x/Koulhs\}}} στην \eqref{eq:1} παίρνουμε θεωρώντας πως ο \textlatin{Koulhs} είναι άντρας :
	\[
		((\neg Father(Koulhs,y) \lor \neg Woman(Koulhs)) \wedge (\neg Mother(Koulhs,y) \lor Woman(Koulhs)) )
	\]
	\noindent
	ή
	\[
		((\neg Father(Koulhs,y) \lor T) \wedge (\neg Mother(Koulhs,y) \lor F) )
	\]
	\noindent
	ή
	\[
		(T) \wedge (\neg Mother(Koulhs,y) )
	\]
	\noindent
	ή
	\begin{equation} \label{eq:3} \tag{5.3}
		\neg Mother(Koulhs,y)
	\end{equation}
	\noindent
	Με αντικατάσταση {\textlatin{ \{y/Volfgang\}}} στην \eqref{eq:3} παίρνουμε:
	\begin{equation} \label{eq:4} \tag{5.4}
		\neg Mother(Koulhs, Volfgang)
	\end{equation}
	\noindent
	Συνεπώς κατλήξαμε πως ο \textlatin{Koulhs} δεν ειναι μητέρα του \textlatin{Volfgang}. Δεχτήκαμε ομως την \eqref{eq:2} η οποία για να ισχύει πρέπει ο 
	\textlatin{Koulhs} να ειναι μητέρα του \textlatin{Volfgang}. Συνεπώς καταλήξαμε σε άτοπο και κατ' επέκταση αποδείξαμε το ζητούμενο δηλαδή:
	\[
		Father(Koulhs,Alex) \rightarrow \neg Mother(Koulhs,Volfgang) 
	\]
	
	\section*{Άσκηση 6}
	
	
	
\end{document}